% Início do preâmbulo
% Classe do documento
\documentclass[12pt]{article}

% Compreender a acentuação feita direto pelo teclado 
% utf8: codificação de caracteres
% Veja: http://www.alessandroduarte.com.br/?page_id=105.
\usepackage[utf8]{inputenc}  

% Idioma: hifenização, nomes em português para capítulos, seções...
\usepackage[brazilian]{babel} 

% muda a condificação da fonte, tornando possível copiar textos com acentos
% Veja: http://wiki.inf.ufpr.br/maziero/lib/exe/fetch.php?media=software:latex-t1-fontenc.pdf
\usepackage[T1]{fontenc}

% estrutura: página e margens
% https://www.overleaf.com/learn/latex/page_size_and_margins
\usepackage[a4paper, top=3cm, right=2cm, left=2cm, bottom = 3cm]{geometry}

% Mudar o algarismo da numeração das páginas (padrão: arábicos)
% Se colocado em qualquer parte do corpo do documento, a numeração a partir daí será iniciada novamente com o novo estilo de numeração declarado.
\pagenumbering{alph} % alph, Alph, roman, Roman, arabic

% Espaçamento entre linhas
\usepackage{setspace} 
\setstretch{1.0} % 1.5 

% Permite a inserção de figuras ou gráficos no texto
\usepackage{graphicx} 


% Permite mudar o estilo do contador no ambiente enumerate
% Opções: A, a (alfabeto);  I, i (romanos); 1 (arábicos)
% Estas opções podem ser combinadas com qualquer string, no entanto, 
% as opções A, a, I, i, 1 devem estar dentro de um {} grupo se
% elas não devam ser considerados especiais.
\usepackage{enumerate}

% Recua a 1a. linha de todas a seções
% Criar parágrafos com ou sem identação/recuo: \indent \noindent
\usepackage{indentfirst} 

% Fixar as imagens/figuras no local em que foram colocadas
\usepackage{float} 

% Fornece recursos para facilitar a escrita de fórmulas matemáticas
% e para melhorar a qualidade tipográfica
\usepackage{amsmath}


% Definindo novos comandos 
\newcommand{\amostra}{$X_1,X_2,\ldots,X_n$}
\newcommand{\real}{\mbox{$I\!\!R$}}
\newcommand{\Real}{I\!\!R}
\newcommand{\bi}{\begin{itemize}}
\newcommand{\ei}{\end{itemize}}
\newcommand{\mat}[1]{\mbox{\boldmath{$#1$}}} % Usar negrito no ambiente matemático
%\renewcommand{\Re}{I\!R} % \Re já é definido pelo LaTeX, para redefini-lo use o \renewcommand


% Informações do documento
\title{Estimação de Parâmetros}
\author{Aluno(s)}
\date{\today{}}

% Estabelece o estilo (cabeçalho e rodapé) da página no documento; mais usual na classe de documento "book". 
%\pagestyle{empty} % cabeçalho e rodapé em branco; sem numeração 
% \pagestyle{plain} % é o estilo padrão: cabeçalho em branco e rodapé com o número da página ao centro
% \pagestyle{headings} % depende da classe do documento
%\pagestyle{myheadings} % o usuário poderá especificar o cabeçalho; usado em conjunto com um dos comandos \markright ou \markboth, este último no caso de documentos da classe "book"
%\markright{\LaTeX}



%%%%%%%%%%%%%%%%%%%%%%%%%%%%%%%%%%%%%%%%%%%%%%%%%%%%%%%%%%%%%%%%%%%%%%
% Fim do preâmbulo
% Início do corpo do documento
\begin{document}

% Inserir as informações do documento
\maketitle

% Estabelece o estilo da página corrente
\thispagestyle{empty}

\section{Introdução}
Os momentos Momento de uma variável aleatória ou de uma distribuição de probabilidade são os valores esperados de potências da variável aleatória. Dois momentos muito utilizados são a esperança e a variância de uma variável aleatória. A esperança é uma medida de posição da distribuição, e a variância, de dispersão dos valores da distribuição de probabilidade. Os momentos podem ser usados para auxiliar a caracterização da distribuição.
Os momentos Momento de uma variável aleatória ou de uma distribuição de probabilidade são os valores esperados de potências da variável aleatória. Dois momentos muito utilizados são a esperança e a variância de uma variável aleatória. A esperança é uma medida de posição da distribuição, e a variância, de dispersão dos valores da distribuição de probabilidade. Os momentos podem ser usados para auxiliar a caracterização da distribuição.
Os momentos Momento de uma variável aleatória ou de uma distribuição de probabilidade são os valores esperados de potências da variável aleatória. Dois momentos muito utilizados são a esperança e a variância de uma variável aleatória. A esperança é uma medida de posição da distribuição, e a variância, de dispersão dos valores da distribuição de probabilidade. Os momentos podem ser usados para auxiliar a caracterização da distribuição.
Os momentos Momento de uma variável aleatória ou de uma distribuição de probabilidade são os valores esperados de potências da variável aleatória. Dois momentos muito utilizados são a esperança e a variância de uma variável aleatória. A esperança é uma medida de posição da distribuição, e a variância, de dispersão dos valores da distribuição de probabilidade. Os momentos podem ser usados para auxiliar a caracterização da distribuição.

\section{Amostragem aleatória simples}
Os momentos Momento de uma variável aleatória ou de uma distribuição de probabilidade são os valores esperados de potências da variável aleatória. Dois momentos muito utilizados são a esperança e a variância de uma variável aleatória. A esperança é uma medida de posição da distribuição, e a variância, de dispersão dos valores da distribuição de probabilidade. Os momentos podem ser usados para auxiliar a caracterização da distribuição.
Os momentos Momento de uma variável aleatória ou de uma distribuição de probabilidade são os valores esperados de potências da variável aleatória. Dois momentos muito utilizados são a esperança e a variância de uma variável aleatória. A esperança é uma medida de posição da distribuição, e a variância, de dispersão dos valores da distribuição de probabilidade. Os momentos podem ser usados para auxiliar a caracterização da distribuição.
Os momentos Momento de uma variável aleatória ou de uma distribuição de probabilidade são os valores esperados de potências da variável aleatória. Dois momentos muito utilizados são a esperança e a variância de uma variável aleatória. A esperança é uma medida de posição da distribuição, e a variância, de dispersão dos valores da distribuição de probabilidade. Os momentos podem ser usados para auxiliar a caracterização da distribuição.
Os momentos Momento de uma variável aleatória ou de uma distribuição de probabilidade são os valores esperados de potências da variável aleatória. Dois momentos muito utilizados são a esperança e a variância de uma variável aleatória. A esperança é uma medida de posição da distribuição, e a variância, de dispersão dos valores da distribuição de probabilidade. Os momentos podem ser usados para auxiliar a caracterização da distribuição.

\section{Propriedades dos estimadores}
Os momentos Momento de uma variável aleatória ou de uma distribuição de probabilidade são os valores esperados de potências da variável aleatória. Dois momentos muito utilizados são a esperança e a variância de uma variável aleatória. A esperança é uma medida de posição da distribuição, e a variância, de dispersão dos valores da distribuição de probabilidade. Os momentos podem ser usados para auxiliar a caracterização da distribuição.

\subsection{Estimadores não tendenciosos}
Os momentos Momento de uma variável aleatória ou de uma distribuição de probabilidade são os valores esperados de potências da variável aleatória. Dois momentos muito utilizados são a esperança e a variância de uma variável aleatória. A esperança é uma medida de posição da distribuição, e a variância, de dispersão dos valores da distribuição de probabilidade. Os momentos podem ser usados para auxiliar a caracterização da distribuição.
Os momentos Momento de uma variável aleatória ou de uma distribuição de probabilidade são os valores esperados de potências da variável aleatória. Dois momentos muito utilizados são a esperança e a variância de uma variável aleatória. A esperança é uma medida de posição da distribuição, e a variância, de dispersão dos valores da distribuição de probabilidade. Os momentos podem ser usados para auxiliar a caracterização da distribuição.


\subsection{Variância de um estimador}
Os momentos Momento de uma variável aleatória ou de uma distribuição de probabilidade são os valores esperados de potências da variável aleatória. Dois momentos muito utilizados são a esperança e a variância de uma variável aleatória. A esperança é uma medida de posição da distribuição, e a variância, de dispersão dos valores da distribuição de probabilidade. Os momentos podem ser usados para auxiliar a caracterização da distribuição.
Os momentos Momento de uma variável aleatória ou de uma distribuição de probabilidade são os valores esperados de potências da variável aleatória. Dois momentos muito utilizados são a esperança e a variância de uma variável aleatória. A esperança é uma medida de posição da distribuição, e a variância, de dispersão dos valores da distribuição de probabilidade. Os momentos podem ser usados para auxiliar a caracterização da distribuição.


%%%%%%%%%%%%%%%%%%%%%%%%%%%%%%%%%%%%%%%%%%%%%%%%%%%%%%%%%%%%%%%%%%%%%%
% Nova página
\newpage 
%\thispagestyle{empty}


% Criar uma seção
\section{Ambiente enumerate}

\noindent \textbf{Exemplo 1}

\begin{enumerate}
    \item método da máxima verossimilhança
    \item método dos momentos
    \item método dos mínimos quadrados
\end{enumerate}


\noindent \textbf{Exemplo 2}
\begin{enumerate} [{exemplo} a)]
    \item método da máxima verossimilhança
    \item método dos momentos
    \item método dos mínimos quadrados
\end{enumerate}


\noindent \textbf{Exemplo 3}
\begin{enumerate} [{A}-1)]
    \item método da máxima verossimilhança
    \item método dos momentos
    \item método dos mínimos quadrados
\end{enumerate}


\noindent \textbf{Exemplo 4}
\begin{enumerate} [A]
    \item método da máxima verossimilhança
    \item método dos momentos
    \item método dos mínimos quadrados
\end{enumerate}


\section{Ambiente itemize}

\noindent \textbf{Exemplo 5}
\begin{itemize}
    \item método da máxima verossimilhança
    \item método dos momentos
    \item método dos mínimos quadrados
\end{itemize}

\noindent \textbf{Exemplo 6}
\begin{itemize}
    \item[$\surd$] método da máxima verossimilhança
    \item[$\surd$] método dos momentos
    \item[$\surd$] método dos mínimos quadrados
\end{itemize}

\noindent \textbf{Exemplo 7}
\begin{itemize}
    \item[M1.] método da máxima verossimilhança
    \item[M2.] método dos momentos
    \item[M3.] método dos mínimos quadrados
\end{itemize}

%%%%%%%%%%%%%%%%%%%%%%%%%%%%%%%%%%%%%%%%%%%%%%%%%%%%%%%%%%%%%%%%%%%%%%
% Nova página
\newpage 

\section{Ambientes e notação matemática}

\noindent \textbf{Notações}
\begin{enumerate}
    \item Subscrito e sobrescrito: $X_1, \ldots, X_n$,  $X_{12}$, $X_{y_a}$, $X^{(a+2)}$, $X^{y^a_b}$, $X_a^b$
	\item Média populacional: $\mu$
	\item Variância populacional: $ \sigma^2 =  \frac{\sum_{i=1}^{N} (x_i-\mu)^2}{N} $ ou $ \sigma^2 =  \dfrac{\sum_{i=1}^{N} (x_i-\mu)^2}{N} $ ou $ \sigma^2 =  \dfrac{\displaystyle\sum_{i=1}^{N} (x_i-\mu)^2}{N}$
	\item Logaritmo de $2x+5$: $\log(2x+5)$
    \item Desigualdades: $x\neq y$, $ x\ge y$, $x \le y$
    \item Na Estatística, usamos a letra grega $\sigma$ para representar uma medida populacional (parâmetro) e denotamos o seu estimador por $\hat{\sigma}$.
    \item Derivadas parciais: $\dfrac{\partial{ f(x,y)}}{\partial{x}}$ ou $\dfrac{\partial{}}{\partial{x}}f(x,y)$
    \item Parênteses, colchetes e chaves: $\left( \dfrac{n-1}{4} \right)$, $\left[ \dfrac{n-1}{4} \right]$,  $\left\{ \dfrac{n-1}{4} \right\}$
    \item Integrais: $\int_{\lambda}^{1} x^2 dx$ ou  $\displaystyle\int_{\lambda}^{1} x^2 dx$
    \item Mínimos e máximos:: $Y=\mbox{min}\{X_1,\ldots,X_n\}$ e $Z=\mbox{máx}\{X_1,\ldots,X_n\}$ (sem o comando mbox, será gerada uma advertência por usar acentos no modo matemático)
\end{enumerate}    

\noindent \textbf{{Ambiente eqnarray:}} alinhar verticalmente uma ou várias equações com ou sem numeração
\begin{eqnarray*} % o asterisco no final do nome do ambiente indica ambiente sem numeração
A & = & B\\
C & = & D\\
E & = & F
\end{eqnarray*}


\begin{eqnarray} % ambiente com numeração em cada expressão
A & = & B\\ 
C & = & D\\ 
E & = & F
\end{eqnarray}


\begin{eqnarray} 
A & = & B\\ 
\nonumber C & = & D\\  % \nonumber retira a numeração da expressão
\nonumber E & = & F
\end{eqnarray}


\noindent \textbf{{Ambiente align:}} alinhar verticalmente uma ou várias equações com ou sem numeração
\begin{align*}
A & = B\\
C & = D\\
E & = F
\end{align*}


\begin{align}
A & = B\\
C & = D\\ 
\nonumber E & = F
\end{align}

\end{document}
