\documentclass[a4paper,12pt]{article} % This defines the style of your paper
\usepackage[brazilian]{babel} 
\usepackage[top = 2.5cm, bottom = 2.5cm, left = 2.5cm, right = 2.5cm]{geometry} 

\usepackage[T1]{fontenc}
\usepackage[utf8]{inputenc}

\usepackage{multirow} 
\usepackage{booktabs}
\usepackage{amsmath}
\usepackage{graphicx} 

\usepackage{setspace}
\setlength{\parindent}{0in}

% Package to place figures where you want them.
\usepackage{float}

% The fancyhdr package let's us create nice headers.
\usepackage{fancyhdr}

% To make our document nice we want a header and number the pages in the footer.

\pagestyle{fancy} % With this command we can customize the header style.

\fancyhf{} % This makes sure we do not have other information in our header or footer.

\lhead{\footnotesize Atividade sobre LaTex}% \lhead puts text in the top left corner. \footnotesize sets our font to a smaller size.

%\rhead works just like \lhead (you can also use \chead)
\rhead{\footnotesize Daniel Magalhães Nunes} %<---- Fill in your lastnames.

% Similar commands work for the footer (\lfoot, \cfoot and \rfoot).
% We want to put our page number in the center.
\cfoot{\footnotesize \thepage} 


	
\begin{document}
	%%%%%%%%%%%%%%%%%%%%%%%%%%%%%%%%%%%%%%%%%%%%%%%%
	% Title section of the document
	%%%%%%%%%%%%%%%%%%%%%%%%%%%%%%%%%%%%%%%%%%%%%%%%
	
	% For the title section we want to reproduce the title section of the Problem Set and add your names.
	
	\thispagestyle{empty} % This command disables the header on the first page. 
	
	\begin{tabular}{p{15.5cm}} % This is a simple tabular environment to align your text nicely 
		{\large \bf CK0033 - INTRODUÇÃO A COMPUTAÇÃO} \\
		 Universidade Federal do Ceará \\ 
		 Daniel Magalhães Nunes, 376163 \\ 2020.2 \\
		\hline % \hline produces horizontal lines.
		\\
	\end{tabular} % Our tabular environment ends here.
	
	\vspace*{0.3cm} % Now we want to add some vertical space in between the line and our title.
	
	\begin{center} % Everything within the center environment is centered.
		{\Large \bf Atividade sobre LaTex} % <---- Don't forget to put in the right number
		\vspace{2mm}
		
	\end{center}  
	
	\vspace{0.4cm}
	
	
	Colocar no LaTeX os exercícios 14, 15 e 16 do capítulo 3 de \cite{MorettinBussab2005}
	
	\begin{enumerate}
		\item[14.] Mostre que:
		\begin{enumerate}
			\item $ \displaystyle\sum_{i=1}^{n} \left( x_i - \overline{x} \right) = 0  $
			\item $ \displaystyle\sum_{i=1}^{n} \left( x_i - \overline{x} \right)^2 = \displaystyle\sum_{i=1}^{n} x_i^2 - n\overline{x}^2 = \displaystyle\sum_{i=1}^{n} x_i^2 - \dfrac{ \left(   \sum x_i \right)^2 }{n}   $
			\item $ \displaystyle\sum_{i=1}^{k} n_i \left( x_i - \overline{x} \right)^2 =  \displaystyle\sum_{i=1}^{n} n_i x_i^2 - n \overline{x}^2$
			\item $ \displaystyle\sum_{i=1}^{k} f_i \left( x_i - \overline{x}  \right)^2 = f_i x_i^2 - \overline{x}^2 $
		\end{enumerate}
	
		\item[15.] Usando os resultados da questão anterior, calcule as variâncias dos problemas  1 e 2 deste capítulo.
		
		\item[16.] Os dados abaixo representam as vendas semanais, em classes de salários mínimos, de vendedores de gêneros alimentícios:
			\begin{table}[H]
				\centering
				\begin{tabular}{c|c}
					\hline
				Vendas semanais	&     Nº de vendedores  \\ \hline
					\multirow{7}{*}{}
					     30 $ \vdash $ 35           &  2
					\multirow{9}{*}{} \\
				35 $ \vdash $ 40	&          10         \\
				40 $ \vdash $ 45	&          18        \\
				45 $ \vdash $ 50	&          50       \\
				50 $ \vdash $ 55	&             70      \\
				55 $ \vdash $ 60	&              30     \\
				60 $ \vdash $ 65	&              18    \\
				65 $ \vdash $ 70	&               2    \\ \hline
				\end{tabular}
			\end{table}
		
		\begin{enumerate}
			\item Faça o histograma das observações.
			\item Calcule a média da amostra, $ \overline{x} $.
			\item Calcule o desvio padrão da amostra, $s$.
			\item Qual a porcentagem das observações compreendidas entre $ \overline{x} - 2s $ e $ \overline{x} + 2s $ ?
			\item Calcule a mediana.
	    \end{enumerate}


	\end{enumerate}
	%%%%%%%%%%%%%%%%%%%%%%%%%%%%%%%%%%%%%%%%%%%%%%%%
	%%%%%%%%%%%%%%%%%%%%%%%%%%%%%%%%%%%%%%%%%%%%%%%%
	\newpage
	\bibliographystyle{apalike}
	\bibliography{bib_latex}
	
	
	
\end{document}
	